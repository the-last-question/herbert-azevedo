\documentclass[conference]{IEEEtran}
\IEEEoverridecommandlockouts
% The preceding line is only needed to identify funding in the first footnote. If that is unneeded, please comment it out.
%----------------------------------------------------------
\usepackage{cite}
\usepackage[pdftex]{graphicx}
% declare the path(s) where your graphic files are
\graphicspath{images/}
\DeclareGraphicsExtensions{.pdf,.jpeg,.png,.jpg}
\usepackage{amsmath,amssymb,amsfonts}
\usepackage{algorithmic}
\usepackage{graphicx}
\usepackage{textcomp}
\usepackage{array}
%\usepackage[caption=false,font=normalsize,labelfont=sf,textfon =sf]{subfig}
\usepackage{dblfloatfix}
\usepackage{url}
\usepackage{lipsum}
\usepackage{listings}
\usepackage{xcolor}
\def\BibTeX{{\rm B\kern-.05em{\sc i\kern-.025em b}\kern-.08em
    T\kern-.1667em\lower.7ex\hbox{E}\kern-.125emX}}
%----------------------------------------------------------
    \lstset{
        escapeinside={/*@}{@*/},
        language=Python,	
        basicstyle=\fontsize{8.5}{12}\selectfont,
        numbers=left,
        numbersep=2pt,    
        xleftmargin=2pt,
        frame=tb,
        columns=fullflexible,
        showstringspaces=false,
        tabsize=4,
        keepspaces=true,
        showtabs=false,
        showspaces=false,
        morekeywords={inline,public,class,private,protected,struct},
        captionpos=b,
        lineskip=-0.4em,
        aboveskip=10pt,
        extendedchars=true,
        breaklines=true,
        prebreak = \raisebox{0ex}[0ex][0ex]{\ensuremath{\hookleftarrow}},
        keywordstyle=\color[rgb]{0,0,1},
        commentstyle=\color[rgb]{0.133,0.545,0.133},
        stringstyle=\color[rgb]{0.627,0.126,0.941},
    }
%----------------------------------------------------------

\begin{document}

\title{Sensor Ultrassônico com Arduino e LCD*\\
{\footnotesize \textsuperscript{*} Sistemas Embarcados: Aluno - herbert.azevedo@aln.senaicimatec.edu.br}

{\footnotesize \textsuperscript{*} Sistemas Embarcados: Prof. Marco Reis - marco.reis@ba.docente.senai.brr}
\thanks{Identify applicable funding agency here. If none, delete this.}
}


\author{\IEEEauthorblockN{1\textsuperscript{st} Given Name Surname}
\IEEEauthorblockA{\textit{dept. name of organization (of Aff.)} \\
\textit{name of organization (of Aff.)}\\
City, Country \\
email address or ORCID}
\and
\IEEEauthorblockN{2\textsuperscript{nd} Given Name Surname}
\IEEEauthorblockA{\textit{dept. name of organization (of Aff.)} \\
\textit{name of organization (of Aff.)}\\
City, Country \\
email address or ORCID}
\and
\IEEEauthorblockN{3\textsuperscript{rd} Given Name Surname}
\IEEEauthorblockA{\textit{dept. name of organization (of Aff.)} \\
\textit{name of organization (of Aff.)}\\
City, Country \\
email address or ORCID}
\and
\IEEEauthorblockN{4\textsuperscript{th} Given Name Surname}
\IEEEauthorblockA{\textit{dept. name of organization (of Aff.)} \\
\textit{name of organization (of Aff.)}\\
City, Country \\
email address or ORCID}
\and
\IEEEauthorblockN{5\textsuperscript{th} Given Name Surname}
\IEEEauthorblockA{\textit{dept. name of organization (of Aff.)} \\
\textit{name of organization (of Aff.)}\\
City, Country \\
email address or ORCID}
\and
\IEEEauthorblockN{6\textsuperscript{th} Given Name Surname}
\IEEEauthorblockA{\textit{dept. name of organization (of Aff.)} \\
\textit{name of organization (of Aff.)}\\
City, Country \\
email address or ORCID}
}

\maketitle

\begin{abstract}
Esse artigo aborda sobre um projeto de um sistema embarcado, que vai fazer com que a informação do valor captado por um sensor de distância ultrassônico(HC-SR04) apareça no LCD 16x2, 
isso vai ocorrer através da comunicação serial entre dois Arduinos Uno R3, onde o Arduino 1 vai estar conectado com o sensor que vai receber o valor da distância e vai passar como 
string para o outro Arduino, que vai estar conectado com o LCD e vai fazer com que a informação apareça no display.
\end{abstract}

\begin{IEEEkeywords}
Arduino, LCD, Sensor, Serial 
\end{IEEEkeywords}

\section{Introduction}
O sistema embarcado é um sistema microprocessado em que um computador está anexado ao sistema que ele controla. 
Um sistema embarcado pode realizar um conjunto de tarefas que foram predefinidas. O sistema é usado para tarefas específicas, e assim, através de engenharia 
é possível otimizar um determinado produto e diminuir o tamanho, bem como os recursos computacionais e o seu valor final.
O primeiro sistema embarcado surgiu em 1961, sendo este o computador guia do míssil nuclear norte-americano, desde então esses sistemas estão sendo aprimorados e 
utilizados em vários aparelhos, como por exemplo aparelhos hospitalares, smatphones, calculadoras e roteadores, que são dispositivos muito utilizados e importantes.
A partir disso, esse projeto foi desenvolvido buscando entender melhor os sistemas embarcados e a plataforma do TinkerCad.

O projeto é sobre a construção de um sistema embarcado na plataforma TinkerCad, onde o sistema vai possuir o arduino UNO R3, o arduino é uma placa de prototipagem eletrônica 
open source que permite o desenvolvimento de vários projeto como por exemplo na automação residencial: apagar as luzes automaticamente, regular a temperatura do ar-condicionado 
e outros, o arduino vai ser utilizado principalmente no transporte de informações e na comunicação serial passando informações do primeiro arduino para o segundo. 
Outro componente utilizado foi o sensor de distância ultrassônico que tem a função de captar a distância de um determinado objeto em uma determinada área e através 
do arduino, com uma função de loop no código e comunicação serial, essa informação da distância vai ficar aparecendo no LCD. Além disso, dentro dessa área que o sensor capta, foi feita
uma divisão em tres áreas, cada uma representada por um led, onde quando um objeto estiver a uma certa distância vai fazer acender um led para mostrar que o objeto está em naquela área.       

\section{Metodologia}



Este projeto foi desenvolvido no ambiente virtual, na plataforma TinkerCad, onde é feito uma simulação de como seria a prototipagem no ambiente presencial, em relação aos 
componentes, tendo vários componentes disponíveis para a construção do protótipo, e as conexões entre os componentes. A plataforma possui também a área do código que proporciona 
a construção de um código para o funcionamento do protótipo, além de disponibilizar várias informações de bibliotecas e outras maneiras de construir o código além do texto. 


\section{Desenvolvimento}


\subsection{Objetivo}\label{AA}
Construir um sistema embarcado, que leia um valor de uma distância fornecido por um sensor e passe esse valor (como string) via comunicação serial para outro
sistema embarcado, fazendo com que esse valor apareça em um LCD.

\subsection{Componentes}
\begin{itemize}
\item 2 Arduino Uno R3;
\item 1 LCD 16 x 2;
\item 3 leds(1 vermelho, 1 verde, 1 amarelo);
\item 4 220 ohms Resistor;
\item 1 Sensor de distância ultrassônico(HC-SR04).
\end{itemize}

\subsection{Conexões}
As conexões feitas no projeto foram a conexão entre os arduinos, arduino 1 e os leds, arduino 1 e sensor de distância ultrassônico e arduino 2 com o LCD, sendo que as conexões foram
feitas utilizando uma placa de ensaio pequena (com exceção da comunicação serial).

A conexão entre os arduinos foi feita para ter a comunicação serial, essa conexão foi simples, basta conectar a entrada rx de um arduino com a entrada tx de outro arduino e a tx de um
arduino com a rx de outro, além disso também tem a conexão entre as entradas GND dos dois arduinos.

A conexão entre o arduino 1 e os leds foi feita conectando a parte do anodo de cada led a 3 portas digitais do arduino 1 e a parte catódica foi conectada a porta GND do arduino 1 através
de uma placa de ensaio pequena e além disso, foi utilizado um resistor de 220 ohms para cada led, evitando que estes queimem, pois, caso o led for ligado direto em uma fonte de 5V sem o 
resistor, o led vai queimar.

A conexão entre o arduino 1 e o sensor de distância ultrassônico foi feita a partir da conexão das portas GND do arduino 1 e do sensor e da conexão do sensor com a fonte de 5V do arduino 1 e
além disso, também ocorreu a conexão das portas echo e trig com duas portas digitais do arduino.

A conexão entre o arduino 2 e o LCD foi feita conectando inicialmente o GND e a fonte, depois a porta do contraste do LCD foi concetada ao GND do arduino 2, seguido de conexões das portas
RS, E, DB4, DB5, DB6 e DB7 com as portas digitais do arduino, e também a conexão da porta de leitura/gravação do LCD com o GND do arduino 2.
Outra conexão feita nessa parte do projeto foi o LED catódico com o GND e o LED anodo com a fonte de 5V do arduino 2, utilizando um resistor de 220 ohms para evitar que o LCD queime.  



\subsection{Codificação}

O código do projeto foi feito no modo de texto da plataforma TinkerCad, onde a linguagem de programação utilizada foi o "C".

O código do primeiro arduino começa definindo quais portas estão os pinos echo e trig do sensor de distância ultrassônico e incluindo as bibliotecas que foram utilizadas 
no decorrer do código, que foram a SoftwareSerial.h e a string.h.
Após isso, foram declaradas as variáveis utilizadas no código, onde as variáveis dos led foram indicadas as portas onde a parte do anodo de cada led estavam conectadas e 
em relação às variáveis float foram atribuídos valores para o desenvolvimento da equação que vai fornecer o valor da distância, foi declarado também as portas rx e tx dos
arduinos para a comunicação serial.
Em seguida foi feito a parte de declarar as quais pinos eram entradas e quais eram saídas e através da função serial.begin, foi feita a configuração da taxa de comunicação da serial
em bits por segundo.
Depois foi feito o loop, onde as funções presentes nesse loop fornecem a distância do sensor e acende o led de acordo com a área que o objeto está localizado, além disso esse loop
possui uma função que converte o valor da distância que está no monitor serial um float para uma string, para que através da comunicação serial seja enviado um valor em forma 
de string para o arduino 2.

O código do segundo arduino começa do mesmo modo que o do primeiro arduino, inicialmente são declaradas as bibliotecas que são as mesmas do primeiro arduino com exceção da 
LiquidCrystal, que é uma biblioteca utilizada no código do display LCD.
Depois, são declaradas as portas tx e rx dos arduinos para a comunicação serial, também foi declarado o valor da distância como string, visto que através
da comunicação serial foi recebido uma string, foram também indicadas as portas digitais do segundo arduino que foram conectadas ao LCD.
Na outra parte do código, foi feito a inicialização do LCD e a configuração da taxa de bits da comunicação serial que foi de 9600, sendo igual a o arduino, pois para a comunicação
entre os arduinos funcionar a taxa de bits tem que ser igual.
A última parte do código foi a parte do loop, onde foram utilizadas funções para que o arduino 2 recebesse as informações do arduino um e transmitisse para o LCD.
As funções utilizadas foram a Serial.available(), que serve para ver se tem alguma informação chegando na comunicação serial, a Serial.readstring(), que tem a função de ler
a string recebida na comunicação, a Serial.flush(), garantindo que o avanço do código só ocorra quando a comunicação serial estiver concluída, lcd.print(), para que o valor da string 
apareça no LCD e lcd.clear(), que limpa as informações anteriores que estão no monitor do LCD. Vale ressaltar, que em algumas parte do código é necessário utilizar a função 
delay(), para o funcionamento dos componentes.


\subsection{Conclusão}\label{SCM}

Nesse artigo foi apresentado um projeto de transmissão de informações entre dois sistemas embarcados, através da comunicação serial, onde um sistema
foi composto por arduino, sensor e leds, e o outro sistema por um arduino e também um LCD.
A partir da realização desse projeto foi possível compreender melhor os sistemas embarcados e entender o funcionamento de vários
componentes como por exemplo o arduino, que possui várias utilidades em muitas áreas na sociedade e também por possuir código aberto (open source), permite a possibilidade de acesso
por qualquer pessoa.  


\end{document}
